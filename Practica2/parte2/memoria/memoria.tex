\documentclass{uc3mpracticas}

\usepackage{helvet}
\usepackage{multicol}
\renewcommand{\familydefault}{\sfdefault}
\usepackage{changepage}
\usepackage{geometry}
\usepackage{caption}
\usepackage{xcolor,colortbl}

\definecolor{Gray}{gray}{0.85}
\definecolor{LightCyan}{rgb}{0.88,1,1}
\definecolor{LightGreen}{rgb}{0.29,1,0.39}


%%%%%%%%%%%%%%%%%%%%%%%%%%%%%%%%%%%%%%%%%%%%%%%%%%%%%%%%%%%%%%%%%%%%%%%%%%%%%%%%
%%%                   Plantilla Prácticas UC3M                               %%%
%%%                Universidad Carlos III de Madrid                          %%%
%%%                   Alejandro Valverde Mahou                               %%%
%%%%%%%%%%%%%%%%%%%%%%%%%%%%%%%%%%%%%%%%%%%%%%%%%%%%%%%%%%%%%%%%%%%%%%%%%%%%%%%%

%Permitir cabeceras y pie de páginas personalizados
\pagestyle{fancy}

%Path por defecto de las imágenes
\graphicspath{ {./images/} }

%Declarar formato de encabezado y pie de página de las páginas del documento
\fancypagestyle{doc}{
  %Cabecera
  \headerpr[1]{Problema de Clasificación: Parte II}{}{Redes de Neuronas Artificiales}
  %Pie de Página
  \footerpr{}{\textbf{UC3M}}{{\thepage} de \pageref{LastPage}}
}

%Declarar formato de encabezado y pie del título e indice
\fancypagestyle{titu}{%
  %Cabecera
  \headerpr{}{}{}
  %Pie de Página
  \footerpr{}{}{}
}


\appto\frontmatter{\pagestyle{titu}}
\appto\mainmatter{\pagestyle{doc}}


\begin{document}
  %Comienzo formato título
  \frontmatter


  %Portada 1 (Centrado todo)
  \centeredtitle{Images/LogoUC3M.png}{Grado en Ingeniería Informática}{Curso 2020/2021}{Redes de Neuronas Artificiales}{Problema de Clasificación: Parte II}{Clasificación de imágenes con Redes Convolucionales}

  \vspace{55mm}

  \authors{Alba Reinders Sánchez}{100383444}{Alejandro Valverde Mahou}{100383383}{}{}{}{}

  \newpage

  %Índice
  \tableofcontents

  \newpage

  %Comienzo formato documento general
  \mainmatter

\section{Introducción}

El problema consiste en clasificar imágenes donde las entradas de la red son directamente los píxeles de cada imagen. Se utiliza el conjunto de datos \textit{CIFAR10}, compuesto por \textbf{60000} imágenes en color (3 canales, \textit{RGB}) de \textbf{32x32} píxeles. El conjunto de datos se divide en 50000 imágenes para entrenamiento y 10000 para test.

\vspace{2mm}

Hay un total de \textbf{10 clases} con 6000 imágenes por clase, por lo que en este caso las clases sí están balanceadas, las diferentes clases son:

\begin{itemize}
  \begin{multicols}{5}
  \item 0 $\rightarrow$ \textit{airplane}
  \item 5 $\rightarrow$ \textit{dog}
  \item 1 $\rightarrow$ \textit{automobile}
  \item 6 $\rightarrow$ \textit{frog}
  \item 2 $\rightarrow$ \textit{bird}
  \item 7 $\rightarrow$ \textit{horse}
  \item 3 $\rightarrow$ \textit{cat}
  \item 8 $\rightarrow$ \textit{ship}
  \item 4 $\rightarrow$ \textit{deer}
  \item 9 $\rightarrow$ \textit{truck}
  \end{multicols}
\end{itemize}


El objetivo de la práctica es entrenar diferentes arquitecturas de \textbf{Perceptrón Multicapa} y \textbf{Redes de Neuronas Convolucionales} para analizar cómo influyen sus hiperparámetros en la resolución del problema de clasificación. Además de comparar sus resultados para comprobar cuál de las dos arquitecturas es más efectiva.



\section{Diseño, entrenamiento y evaluación del PM}

Estas dos secciones deben contener una breve  descripción  de  los  experimentos  realizados,  los  resultados  obtenidos  y  su análisis.

\section{Diseño, entrenamiento y evaluación de la CNN}



\section{Comparación PM y CNN}




\section{Conclusión}









\end{document}
