\documentclass{uc3mpracticas}

\usepackage{helvet}
\usepackage{multicol}
\renewcommand{\familydefault}{\sfdefault}


%%%%%%%%%%%%%%%%%%%%%%%%%%%%%%%%%%%%%%%%%%%%%%%%%%%%%%%%%%%%%%%%%%%%%%%%%%%%%%%%
%%%                   Plantilla Prácticas UC3M                               %%%
%%%                Universidad Carlos III de Madrid                          %%%
%%%                   Alejandro Valverde Mahou                               %%%
%%%%%%%%%%%%%%%%%%%%%%%%%%%%%%%%%%%%%%%%%%%%%%%%%%%%%%%%%%%%%%%%%%%%%%%%%%%%%%%%

%Permitir cabeceras y pie de páginas personalizados
\pagestyle{fancy}

%Path por defecto de las imágenes
\graphicspath{ {./images/} }

%Declarar formato de encabezado y pie de página de las páginas del documento
\fancypagestyle{doc}{
  %Cabecera
  \headerpr[1]{Problema de Clasificación: Parte I}{}{Redes de Neuronas Artificiales}
  %Pie de Página
  \footerpr{}{\textbf{UC3M}}{{\thepage} de \pageref{LastPage}}
}

%Declarar formato de encabezado y pie del título e indice
\fancypagestyle{titu}{%
  %Cabecera
  \headerpr{}{}{}
  %Pie de Página
  \footerpr{}{}{}
}


\appto\frontmatter{\pagestyle{titu}}
\appto\mainmatter{\pagestyle{doc}}


\begin{document}
  %Comienzo formato título
  \frontmatter


  %Portada 1 (Centrado todo)
  \centeredtitle{Images/LogoUC3M.png}{Grado en Ingeniería Informática}{Curso 2020/2021}{Redes de Neuronas Artificiales}{Problema de Clasificación: Parte I}{Clasificación de imágenes del cielo con el Perceptrón Multicapa}

  \vspace{55mm}

  \authors{Alba Reinders Sánchez}{100383444}{Alejandro Valverde Mahou}{100383383}{}{}{}{}

  \newpage

  %Índice
  \tableofcontents

  \newpage

  %Comienzo formato documento general
  \mainmatter

\section{Introducción}

El problema planteado consiste en clasificar imágenes del cielo con el objetivo de ayudar a la estimación de la radiación solar que incide en un lugar, ya que la cantidad de esta radiación varía si hay nubes o no, y del tipo de las nubes.

\vspace{3mm}

Para ello, se hace una simplificación del problema real reduciendo a tres posibles clases las imágenes:

\begin{itemize}
  \item Cielo Despejado
  \item Nube \textit{(sólo un tipo de nube)}
  \item Multinube \textit{(varios tipos de nube)}
\end{itemize}

Dado que en esta primera parte de la práctica se trabaja con un \textbf{Perceptrón Multicapa}, es necesario transformar la información de la imagen en distintos atributos numéricos.

\vspace{3mm}

Los datos que se usan en esta práctica provienen del \textit{grupo MATRAS de la Universidad de Jaén} y contienen estadísticos y transformaciones de imágenes del cielo completo, que permiten al Perceptrón Multicapa realizar la clasificación.

\vspace{2mm}

El conjunto de datos tiene \textbf{717} instancias con \textbf{12} atributos de entrada y \textbf{1} atributo de salida, la clase.

\begin{enumerate}
  \begin{multicols}{2}
  \item Media del canal azul
  \item Media del canal rojo
  \item Desviación típica del canal azul
  \item Sesgo del canal azul
  \item Diferencia de medias Rojo – Verde
  \item Diferencia de medias Rojo – Azul
  \columnbreak
  \item Diferencia de medias Verde – Azul
  \item Entropía del canal azul
  \item Energía del canal azul
  \item Contraste del canal azul
  \item Homogeneidad del canal azul
  \item Cobertura
  \end{multicols}
\end{enumerate}



\section{Preparación de los datos}

Se debe hacer un preprocesado de los datos para poder realizar de manera más efectiva el entrenamiento de la red. Primero se normaliza el conjunto de datos y después se divide en 4 subconjuntos, ya que se lleva a cabo \textit{validación cruzada estratificada} de 4 hojas.

\subsection{Normalizar}

Es necesario normalizar los datos de entrada, dado que cada atributo suele tener rangos de valores muy diferentes. Normalizarlos en el intervalo [0,1] evita posibles sesgos generados por esta difencia de rangos de los atributos durante el aprendizaje de la red.


\subsection{Preparar los datos para Validación Cruzada}


\section{Experimentación}

\section{Análisis de los resultados}

Evolución de los errores de entrenamiento y test a lo largo del aprendizaje solo para algunos (o algún) de los experimentos (los más significativos).


Porcentajes de aciertos de entrenamiento y test al finalizar el aprendizaje de todos los experimentos realizados.

\section{Conclusión}


\end{document}
